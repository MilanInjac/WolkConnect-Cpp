The goal of this document is to provide a high-\/level overview of Wolk\+Connect libraries.

Wolk\+Connect libraries are used to enable a device’s communication with Wolk\+About IoT Platform. Using Wolk\+Connect libraries in the software or firmware of a device will drastically decrease the time to market for developers or anyone wanting to integrate their own product with Wolk\+About IoT Platform.

The available Wolk\+Connect libraries (C, C++, Java, Python) are platform independent, with a special note that the Wolk\+Connect-\/C library is suitable to be adapted for use on bare-\/metal devices.

Features of Wolk\+About IoT Platform that have been incorporated into Wolk\+Connect libraries will be disambiguated with information on how to perform these features on devices by using Wolk\+Connect’s A\+PI. Conception The intention of the Wolk\+Connect library is to be used as a dependency in other firmwares or softwares that have their own existing business logic. Wolk\+Connect library is not by any means a single service to control the device.

Using Wolk\+Connect library does not require knowledge of Wolk\+About IoT Platform. The user only utilises A\+P\+Is provided by the User Application Layer, thereby reducing time-\/to-\/market required.

Providing Wolk\+Connect library with IP connectivity from the Hardware Abstraction Layer is expected from the user.

The gray section of Fig.\+1.\+1 represents the developer’s software/firmware in a programing language supported by Wolk\+Connect library

Fig.\+1.\+1 Wolk\+Connect library represented in general software/firmware architecture

Ovde kreni sada u opis samog konektora, iz cega se on sastoji. Imas 4 celine, pa mozda onda ova slika 1.\+1. Kazi da je sva komunikacija kriptovana.

A common dependency for Wolk\+Connect libraries is a J\+S\+ON library used for parsing data that is exchanged with Wolk\+About IoT Platform. This data is formatted using a custom J\+S\+ON based protocol defined by Wolk\+About IoT Platform.

Communication between Wolk\+Connect library and Wolk\+About IoT Platform is achieved through the use of the M\+Q\+TT protocol. Wolk\+Connect libraries have another common dependency, an implementation of an M\+Q\+TT client that will exchange data with an M\+Q\+TT server that is part of Wolk\+About IoT Platform. The communication between Wolk\+Connect library and Wolk\+About IoT Platform is made secure with the use of Transport Layer Security (T\+LS).

Wolk\+Connect libraries use network connectivity provided by the OS, but on devices where these calls are not available it is the user’s responsibility to provide implementations for opening a socket and send/receive methods to the socket.

Picture 1.\+1. Wolk\+Connect library layers

All the business logic has to be implemented by the user, Wolk\+Connect libraries should simply be considered a third party dependency to enable communication to Wolk\+About IoT Platform. Connect and disconnect

Every connection from a Wolk\+Connect library to Wolk\+About IoT Platform is authenticated with a device key and a device password. These credentials are created on Wolk\+About IoT Platform when a device is created and are unique to that device. Only one active connection is allowed per device. Attempting to create an additional connection with the same device credentials will terminate the previous connection. The connection is made secure by default in all Wolk\+Connect libraries through the use of Transport Layer Security (T\+LS). Connecting without T\+LS is possible, refer to a specific Wolk\+Connect library’s connect method for more information.

Disconnecting will gracefully terminate the connection, but in the cases of ungraceful disconnections, eg. due to a networking error, a last will message will be broadcast. This last will message will be responsible for immediately declaring the device offline on Wolk\+About IoT Platform. Device

Wolk\+Connect libraries and Wolk\+About IoT Platform separate a device’s functionality into two distinct parts\+: Device data -\/ represents information that is being communicated to Wolk\+About IoT Platform Device management -\/ dictates the behaviour of how the device operates Device Data

Real world devices can perform a wide variety of operations that result in meaningful data. These operations could be to conduct a measurement, monitor a certain condition or execute some form of command. The data resulting from these operations have been modeled into three distinct types of data on Wolk\+About IoT Platform\+: sensors, alarms, and actuators.

Information needs to be distinguishable, so every piece of data sent from the device needs to have an identifier. This identifier is called a reference, and all references of a device must be unique. Sensors

Sensor readings are stored on the device before explicitly being published to Wolk\+About IoT Platform. If the exact time when the reading occured is meaningful information, it can be assigned to the reading as a U\+TC timestamp. If this timestamp is not provided, Wolk\+About IoT Platform will assign the reading a timestamp when it has been received, treating the reading like it occured the moment it arrived.

Readings could be of a very high precision, and although this might not be fully displayed on the dashboard, the information is not lost and can be viewed on different parts of Wolk\+About IoT Platform.

Sensors readings like G\+PS and accelerometers hold more than one single information and these types of readings are supported in Wolk\+Connect libraries and on Wolk\+About IoT Platform. See documentation of a specific Wolk\+Connect library on how to create multi-\/value readings in different programming languages. Alarms

Alarms are derived from some data on the device and are used to indicate the state of a condition, so the alarm value can either be on or off. Like sensor readings, alarm messages are stored on the device before being published to Wolk\+About IoT Platform. Alarms can also have a U\+TC timestamp to denote when the alarm occurred, but if the timestamp is omitted then Wolk\+About IoT Platform will assign a timestamp when it receives the alarm message.

Actuators

Actuators are used to enable Wolk\+About IoT Platform to set the state of some part of the device, eg. flip a switch or change the gear of a motor. In order to achieve this, Wolkabout IoT Platform needs to be aware of what the current state of the actuator is. Here, it is the user’s responsibility to implement an actuator status provider that will report the current value and state of the actuator to Wolk\+About IoT Platform.

The possible actuator states are\+:

R\+E\+A\+DY -\/ waiting to receive a command to change its value B\+U\+SY -\/ in the process of changing its value E\+R\+R\+OR

Now that the device has notified Wolk\+About IoT Platform about the current state of its actuators, it needs to be able to listen for commands issued from Wolk\+About IoT Platform. First, a list of actuator references that are available on the device needs to be provided when connecting to Wolk\+About IoT Platform in order to subscribe to messages containing actuation commands. Second, the user has to implement an actuation handler that will execute the commands that have been issued from Wolk\+About IoT Platform.

To summarise, when an actuation command is issued from Wolk\+About IoT Platform, it will be passed to the actuation handler that will attempt to execute the command, and then the actuator status provider will report back to Wolk\+About IoT Platform with the current value and state of the actuator.

Publishing of actuator statuses is performed immediately, but if the device is unable to publish then the information will be stored on the device until the next successful publish. Device Management

Keep Alive Mechanism

In cases where the device is connected to the platform but is not publishing any data for prolonged periods of time, the device may be declared offline. This is especially true for devices that only have actuators for example. To prevent this issue, a keep alive mechanism will periodically send a message to Wolk\+About IoT Platform. This mechanism can also be disabled to reduce bandwidth usage.

Configuration Options

Configuration options are enabled on the device by the user’s implementation of a configuration provider and a configuration handler. All configuration options are sent initially from the device as a single information and after that new configuration values can be issued from Wolk\+About IoT Platform. Configuration options are always sent as a whole, even when only one value changes.

Device Firmware Update

Wolk\+About IoT Platform gives the possibility of updating the firmware of a device. In order to update the firmware, the user must create a firmware handler. This firmware handler will specify the following parameters\+: current firmware version, desired size of firmware chunk to be received from Wolk\+About IoT Platform, maximum supported firmware file size, download location, implementation of a firmware installer that will be responsible for the installation process. Optionally, an implementation of firmware download handler that will download a file from an U\+RL issued from Wolk\+About IoT Platform. 